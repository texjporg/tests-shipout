\ifdefined \AddToHook
  % plautopatch/issues/12
  % platex-tools/pull/16
\else
  \RequirePackage{plautopatch}% pxpgfrcs, pxeveryshi
\fi
\documentclass[dvipdfmx,twocolumn]{tarticle}
\setlength\columnseprule{.4pt}
\pagestyle{empty}
\usepackage{tikz}

\usetikzlibrary{shadings,shadows,fadings,calendar}

\begin{document}

\section{Test of \texttt{fadings} library}%%% from http://www.texample.net/tikz/examples/coffee-cup/

\tikzfading[name=fade out,
 inner color=transparent!0, outer color=transparent!100]

\mbox{\yoko
\begin{tikzpicture}
 \foreach \c [count=\i from 0] in {white,red!75!black,blue!25,orange}{

   \tikzset{xshift={mod(\i,2)*3cm}, yshift=-floor(\i/2)*3cm}
   \colorlet{cup}{\c}

   % Saucer
   \begin{scope}[shift={(0,-1-1/16)}]
     \fill [black!87.5, path fading=fade out]
       (0,-2/8) ellipse [x radius=6/4, y radius=3/4];
     \fill [cup, postaction={left color=black, right color=white, opacity=1/3}]
       (0,0) ++(180:5/4) arc (180:360:5/4 and 5/8+1/16);
     \fill [cup, postaction={left color=black!50, right color=white, opacity=1/3}]
       (0,0) ellipse [x radius=5/4, y radius=5/8];
     \fill [cup, postaction={left color=white, right color=black, opacity=1/3}]
       (0,1/16) ellipse [x radius=5/4/2, y radius=5/8/2];
     \fill [cup, postaction={left color=black, right color=white, opacity=1/3}]
       (0,0) ellipse [x radius=5/4/2-1/16, y radius=5/8/2-1/16];
   \end{scope}

   % Handle
   \begin{scope}[shift=(10:7/8), rotate=-30, yslant=1/2, xslant=-1/8]
     \fill [cup, postaction={top color=black, bottom color=white, opacity=1/3}]
       (0,0) arc (130:-100:3/8 and 1/2) -- ++(0,1/4) arc (-100:130:1/8 and 1/4)
       -- cycle;
     \fill [cup, postaction={top color=white, bottom color=black, opacity=1/3}]
       (0,0) arc (130:-100:3/8 and 1/2) -- ++(0,1/32) arc (-100:130:1/4 and 1/3)
       -- cycle;
   \end{scope}

   % Cup
   \fill [cup!25!black, path fading=fade out]
     (0,-1-1/16) ellipse [x radius=3/4, y radius=1/3];
   \fill [cup, postaction={left color=black, right color=white, opacity=1/3/2},
     postaction={bottom color=black, top color=white, opacity=1/3/2}]
     (-1,0) arc (180:360:1 and 5/4);
   \fill [cup, postaction={left color=white, right color=black, opacity=1/3}]
     (0,0) ellipse [x radius=1, y radius=1/2];
   \fill [cup, postaction={left color=black, right color=white, opacity=1/3/2},
     postaction={bottom color=black, top color=white, opacity=1/3/2}]
     (0,0) ellipse [x radius=1-1/16, y radius=1/2-1/16];

   % Coffee
   \begin{scope}
     \clip ellipse [x radius=1-1/16, y radius=1/2-1/16];
     \fill [brown!25!black]
       (0,-1/4) ellipse [x radius=3/4, y radius=3/8];
     \fill [brown!50!black, path fading=fade out]
       (0,-1/4) ellipse [x radius=3/4, y radius=3/8];
   \end{scope}
 }
\end{tikzpicture}}


\section{Test of \texttt{shadows} library}%%% from http://www.texample.net/tikz/examples/overlapping-arrows/

\definecolor{darkblue}{rgb}{0.2,0.2,0.6}
\definecolor{darkred}{rgb}{0.6,0.1,0.1}
\definecolor{darkgreen}{rgb}{0.2,0.6,0.2}

\def\arrow{
 (10.75:1.1) -- (6.5:1) arc (6.25:120:1) [rounded corners=0.5] --
 (120:0.9) [rounded corners=1] -- (130:1.1) [rounded corners=0.5] --
 (120:1.3) [sharp corners] -- (120:1.2) arc (120:5.25:1.2)
 [rounded corners=1] -- (10.75:1.1) -- (6.5:1) -- cycle
}

\tikzset{
 ashadow/.style={opacity=.25, shadow xshift=0.07, shadow yshift=-0.07},
}

\def\arrows[#1]{        
 \begin{scope}[scale=#1]
   \draw[color=darkred, %
   drop shadow={ashadow, color=red!60!black}] \arrow;

   \draw[color=darkgreen, bottom color=green!90!black, top color=green!60, %
   drop shadow={ashadow, color=green!60!black}] [rotate=120] \arrow;

   \draw[color=darkblue, right color=blue, left color=blue!60, %
   drop shadow={ashadow, color=blue!60!black}] [rotate=240] \arrow;

   % to hide the green shadow
   \draw[color=darkred, left color=red, right color=red!60] \arrow;
 \end{scope}
}

\mbox{\yoko
\begin{tikzpicture}
 \arrows[2.5];
\end{tikzpicture}}

\section{Test of \texttt{shadings} library}%%% from http://www.texample.net/tikz/examples/calendar-circles/

\colorlet{winter}{blue}
\colorlet{spring}{green!60!black}
\colorlet{summer}{orange}
\colorlet{fall}{red}
\newcount\mycount

\mbox{\yoko
\begin{tikzpicture}[transform shape,
 every day/.style={anchor=mid,font=\tiny}]
 \node[circle,shading=radial,outer color=blue!30,inner color=white,
   minimum width=15cm] {\textcolor{blue!80!black}{\Huge\the\year}};
 \foreach \month/\monthcolor in
   {1/winter,2/winter,3/spring,4/spring,5/spring,6/summer,
   7/summer,8/summer,9/fall,10/fall,11/fall,12/winter} {
     % Computer angle:
     \mycount=\month
     \advance\mycount by -1
     \multiply\mycount by 30
     \advance\mycount by -90
     \shadedraw[shading=radial,outer color=\monthcolor!30,middle color=white,
       inner color=white,draw=none] (\the\mycount:5.4cm) circle(1.4cm);
     % The actual calendar
     \calendar at (\the\mycount:5.4cm) [
         dates=\the\year-\month-01 to \the\year-\month-last]
     if (day of month=1) {\large\color{\monthcolor!50!black}\tikzmonthcode}
     if (Sunday) [red]
     if (all) {
     % Again, compute angle
     \mycount=1
     \advance\mycount by -\pgfcalendarcurrentday
     \multiply\mycount by 11
     \advance\mycount by 90
     \pgftransformshift{\pgfpointpolar{\mycount}{1.2cm}}};}
\end{tikzpicture}}

\end{document}
